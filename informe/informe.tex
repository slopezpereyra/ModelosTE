\documentclass[a4paper, 12pt]{article}

\usepackage{parskip}
\usepackage[utf8]{inputenc}
\usepackage[T1]{fontenc}
\usepackage{textcomp}
\usepackage{amssymb}
\usepackage{newtxtext} \usepackage{newtxmath}
\usepackage{amsmath, amssymb}
\newtheorem{problem}{Problem}
\newtheorem{example}{Example}
\newtheorem{lemma}{Lemma}
\newtheorem{theorem}{Theorem}
\newtheorem{problem}{Problem}
\newtheorem{example}{Example} \newtheorem{definition}{Definition}
\newtheorem{lemma}{Lemma}
\newtheorem{theorem}{Theorem}
\DeclareMathAlphabet{\mathcal}{OMS}{cmsy}{m}{n}

\begin{document}

\begin{titlepage}
   \begin{center}
       \vspace*{1cm}

       \Huge
       \textbf{Modelos y simulación - Trabajo especial}

       \vspace{0.5cm}
        FAMAF - UNC
            
       \vspace{1.5cm}
       \large
       \textbf{Juliana García - Santiago López Pereyra}
       \normalsize

       \vfill
            
            
     
   \end{center}
\end{titlepage}

   
\maketitle
\pagebreak

\section{Descripción del problema}

Sea 

\begin{equation}
    \lambda(t) = 20 + 10\cos \left( \frac{\pi t}{12} \right) 
\end{equation}

\subsection{Caracterizando propiedades del sistema}

Como el coseno oscila en $[-1, 1]$, $\lambda(t)$ tiene máximo 30 y mínimo 10. Más
aún, $\pi\left( \frac{\pi t}{12} \right) $ completa un ciclo cuando $\pi t / 12
= 2 \pi \iff t = 24$. Se sigue que en $t = 12$ alcanza su mínimo (mitad del
ciclo recorrido). 

Nos interesa caracterizar los períodos donde el servidor tendrá mayor y menor
actividad. Los caracterizaremos como las regiones de $t$ en que $\lambda(t)$
está por encima y por debajo de su punto medio, respectivamente.
No es difícil ver que $\lambda(t) > 20 \iff \cos(\pi t / 12) > 0$. Pero el
coseno es positivo si su argumento pertenece a $[- \pi / 2, \pi / 2]$. Por ende, 

\begin{align}
    \lambda(t) > 20 &\iff -\pi / 2 + 2k\pi \leq \frac{\pi t}{12}  \leq \pi / 2 +
    2k \pi\\ 
                    &\iff-6 + 24k \leq t \leq 6 + 24k
\end{align}

Si restringimos $t \in [0, 48]$, esto vale si y solo si 

\begin{equation}
    t \in (0, 6) \cup  (18, 30) \cup  (42, 48)
\end{equation}

El complemento de este conjunto sobre el universo $[0, 48]$ nos da los periodos
de menor actividad.
~
El valor medio de llegadas en las 48 horas es:

\begin{equation}
\int_0^{48} \lambda(t) ~ dt = \int_0^1 
20 + 10 \cos(\frac{\pi
    t}{12})
    ~ dt = 960
\end{equation}

Esto implica que $\frac{960}{48} = 20$ es el valor medio de llegadas por hora.
Incluso en períodos de máxima actividad, la cantidad esperada de llegadas por
hora es prácticamente la misma:

\begin{equation}
    \frac{1}{6}\int_0^{6}\lambda(t) ~ dt = 21
\end{equation}

Como se atiende $35$ personas por hora, esto significa que incluso en los
períodos de mayor actividad se espera que el servidor atienda a todas las
personas.
































\end{document}



